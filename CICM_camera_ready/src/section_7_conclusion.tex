

\section{Conclusion}

We have developed an image-based method for solving geometric construction problems. The method builds on the \maskrcnn visual recognizer, which is adapted to predict the next step of a geometric construction given the current state of the construction, and further combined with a tree search mechanism to explore the space of possible constructions. To train and test the method, we have used Euclidea, a construction game with geometric problems with an increasing difficulty. To train the model, we have created a data generator that generates new configurations of the Euclidea constructions. 

In a supervised setting, the method learns to solve all 68 kinds of geometric construction problems from the first six level packs of Euclidea with an average 92\% accuracy. When evaluated on new kinds of problems unseen at training, which is a significantly more challenging set-up, our method solves 31 of the 68 kinds of Euclidea problems. 
To solve the unseen problems our model currently relies on having seen a similar (or more complex) problem at training time. Solving unseen problems that are more complex than those seen at training remains an open challenge. Addressing this challenge is likely going to require developing new techniques to efficiently explore the space of constructions as well as mechanisms to learn from successfully completed constructions, a set-up similar to reinforcement learning. 

Although in this paper we focus on synthetically generated data, our results open up the possibility to solve even hand-drawn geometric problems if corresponding training data is provided. 
In real-world settings, descriptions and discussions of problems in math, physics and other sciences often contain an image or a drawing component. Diagrams and illustrations are also an essential part of real-world technical documentation (e.g.~patents). The ability to automatically identify and combine visually defined geometrical primitives into more complex patterns, as done in our work, is a step towards automatic systems that can identify compositions of geometric patterns in technical documentation. Think, for example, of an ``automatic patent lawyer assistant''.


